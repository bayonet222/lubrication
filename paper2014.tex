 

\begin{longtable}[l]{p{50pt} p{400pt} }
\textbf{Symbol} &  \textbf{Description}  \\
$a,b$ & the pad occupies the region $a < x < b$ \\
$d$ & texture depth \\
$d_{\mbox{\scriptsize lub}}$ & lubricant film thickness at inlet boundary \\
$f$ & instanteneous friction coefficient \\
$\overline{f}$ & time-averaged friction coefficient \\
$g$ & function controlling the onset of friction in the Couette term \\
$h$ & clearance between runner and pad (in the $z$ direction) \\
$h_L$ & function describing the texture shape \\
$h_U$ & function characterizing the pad profile  \\
$m$ & pad linear mass per unit width\\
$p$ & hydrodynamic pressure \\
$t$ & time variable\\
$u$ & relative velocity between the pad and the runner \\
$x$, $z$ & coordinate directions  \\
$x_{\ell}$,$x_{r}$ & left and right boundaries of the computational domain  \\
$L$ & pad length\\
$C$ & minimum clearance \\
$F$ & friction force per unit width\\
$H$ & non-dimensionalization length for $z$, $Z$ and $e$ \\
$R$ & curvature radius of the pad's profile \\
$U$ & non-dimensionalization for the relative velocity $u$ \\
$W^{a}$ & applied load per unit width\\
$W^{h}$ & hydrodynamic force per unit width\\
$Z$ & pad position in the $z$ direction \\
$\theta$ & saturation variable \\
$\theta_s$ & threshold for the onset of friction in the Couette term \\
$\lambda$ & period of textures  \\
$\mu$ & dynamic viscosity \\
$\Sigma$ & cavitation boundary \\
$\Omega^0$ & cavitated region \\
$\Omega^+$ & pressurized region \\
\end{longtable}

\setcounter{table}{0} 

\section{INTRODUCTION}

Textured tribological surfaces have attracted much attention of
the research community lately. After a significant number of
experimental and theoretical studies, the possibility of
reducing friction by means of microtextures has been established,
together with a basic understanding of why this happens (at
least in the hydrodynamic regime) \cite{dobrica2010,gad2012,etsion2013,checo13}.

That certain textures reduce friction in 
some hydrodynamic bearings is by no means an
obvious phenomenon. Buscaglia et al \cite{buscaglia05c,buscaglia07} performed
asymptotic analyses of general smooth (i.e., untextured) surfaces 
by introducing short-wavelength
periodic perturbations of arbitrary shape and obtained
that the untextured shape {\em always} maximizes the load 
carrying capacity and minimizes the friction coefficient.
This implies that, for friction reduction to take place,
there exist two possibilities: It can be a consequence 
of a {\em finite} perturbation (outside the validity of
asymptotic theory),
{\em or} involve physical mechanisms which were
not considered in the aforementioned mathematical studies, such as
{\em cavitation}. 

In a recent study, Checo et al \cite{checo13} discussed
several hundred numerical simulations and concluded that
in fact {\em both} of the previous possibilities hold
true in textured bearings that exhibit less friction than
their untextured counterparts. As had already been
advanced by Etsion \cite{etsion2013}, friction reduction
only occurs in high-conformity bearings, in which the
surfaces are so parallel that a texture of some suitable
size and depth manages to produce local cavitation.

High-conformity bearings are not unfrequent in technology.
Assuming a bearing of length $L$ with a nominally planar surface
opposing a surface with curvature radius $R$, the degree
of conformity can be measured by the quotient $R/L$. For
the compression ring of an internal combustion
engine $R/L$ is already quite high (about $40$) \cite{gad2012}, 
and it is much higher for other piston rings 
(oil rings in particular) and for other contacts such as
seals.

It was numerically shown by Gadeschi et al \cite{gad2012} with a
non-mass-conserving model, and later by Checo et al \cite{checo13}
with a mass-conserving model, that for a given load the
minimal friction is in general obtained with a moderate
$R/L$ ratio (of the order of 10). 
Further, their results indicate that  moderate-conformity contacts
cannot be improved by texturing the surfaces.

Their numerical findings also show, on the other hand, that high-conformity
contacts can indeed be improved by texturing. The friction cannot
be reduced to the point of becoming smaller than that of
smooth bearings with moderate conformity, but nevertheless
significant reduction appears to be achievable. These findings
are consistent with pioneering experimental results of Costa \& Hutchins \cite{costa2007}, and with more recent studies by Kovalchenko et al \cite{koval2011},
Yin et al \cite{yin2012}, Tomanik \cite{tomanik2013}, Scaraggi et al 
\cite{scaraggi2013}, among others\cite{qiu11,grabon2013,cross2013,zhang2012}.

Friction-reducing texturing is especially
practical when the texture is placed on the surface that
exhibits less wear (e.g.; the runner of a thrust bearing, the
cylinder liner of a ring/liner contact \cite{tomanik2013}). The reason for this is
that, since the curvature
radius $R$ increases with wear, the texture (if undamaged)
becomes more beneficial as time evolves.

However, though the general picture seems to be coming to
a consensus, little is still known about the actual physical
mechanisms that take place at a cavitating microtextured
contact. Detailed quantitative assessments that allow for
the practical selection of a suitable texture for a
specific device are also very scarce.

For these reasons, in this article we go back to the basics and
study the arguably simplest bearing: an infinitely-wide
thrust bearing with a circular-arc-shaped pad. This bearing
is assumed sliding on
a runner with the arguably simplest texture: a sinusoidal one.
The study is conducted with Elrod-Adams model, which though
not free of criticism \cite{ausas2013, buscaglia2013},
is the most accepted model for hydrodynamic lubrication with
cavitation. This model is mass-conserving, a property that as shown
by Ausas et al \cite{ausas07} is crucial for obtaining realistic
results when simulating textured bearings. The finite volume
code that implements the model is thoroughly described by
Ausas et al \cite{ausas09}.

The plan of this paper is as follows: In Section 2 we briefly
recall the Elrod-Adams model and introduce the problem under consideration
and its non-dimensionalization. Section 3 discusses the numerical
aspects of the work and the general setting of the computational
experiments. Section 4 is devoted to reporting and discussing the
numerical results, by first providing detailed analyses of three
representative cases and then showing extensive charts of
friction coefficient and clearance as functions of the
problem parameters. The basic mechanism
of friction reduction identified in this work is a local 
pressurization of the convergent
microwedges at each texture cell, accompanied by local cavitation
at the divergent microwedges. The cavitation bubble that forms
at each texture cell is crucial for preventing the appearance of
local negative pressure peaks that would cancel out the positive
lift force generated at the convergent microwedges. This is
consistent with mechanisms suggested by other authors 
\cite{fowell07,gad2012}, and puts forward
the importance of cavitation bubbles. 
Section 5 is devoted to a thorough discussion of the numerical
results, identifying and interpreting trends in the predictions
of the model. Concluding remarks are left for Section 6.

\section{MODEL}


As shown in Fig. \ref{fig:domain}, the direction of the motion of the pad is chosen to be $x$, parallel to the runner surface. 
The textures on the runner are given by a function $z=-h_L(x)\leq 0$ ($h_L(x)$=0 for the untextured liner).

\begin{figure}
    \begin{center}
	{\scalebox{1.0}{\input{figs/domain_new.pdf_t}}}
   \end{center}
\caption{Scheme of a the domain along the pad's 
direction of motion, with the forces acting on it.}
\label{fig:domain}
\end{figure}


A single pad with a profile $h_U(x)$ is assumed sliding with constant
velocity $-u$ against the runner. The pad is free to move vertically under the
applied forces. The profile $h_U(x)$ 
satisfies 
$$\min h_U(x)=0$$ 
and its analytical expression corresponds to an arc of circumference 
of radius $R$. 
The computational domain is attached to the pad, which is assumed to be
of length $L$ and to extend between $x=a$ and $x=b$. The runner thus moves
in the positive $x$ direction with constant velocity $u$.
Outside of the pad's location the gap between 
the surfaces is assumed uniform of thickness $e$, large enough not to
affect the results. The (artificial)
simulation boundaries are located at $x=x_\ell<a$ and
$x=x_r > b$. 



Under the assumptions above, the gap between the pad
and the runner is given by
\begin{equation}
h(x,t)= \left \{
\begin{array}{ll}
h_L(x-u\,t)+h_U(x)+Z(t)&\mbox{if}~~a < x < b\\
h_L(x-u\,t)+e& \mbox{otherwise}
\end{array} \right.
\end{equation}
where $Z(t)$ denotes the vertical distance between the pad and the
$z$=0 line. The instantaneous {\em clearance} $C(t)$ is defined as
$$
C(t) ~=~ \min_{x\,\in\,(a,b)} ~h(x,t)
$$

\subsection{Hydrodynamics and cavitation modeling}

We adopt the well-known Elrod-Adams model \cite{elrod1}, which
incorporates into a single formulation Reynolds equation for the
pressurized region and Jacobsson-Floberg-Olsson boundary conditions.
This model is mass conserving, which  
is essential for obtaining physically meaningful results
in lubrication problems involving textured surfaces \cite{ausas07}.


The model postulates the computation of two fields, $p$ and $\theta$,
which correspond to the hydrodynamic pressure and to an auxiliary
saturation-like variable, respectively, that (weakly) satisfy the equation
\begin{equation}
\frac{\partial }{\partial x} \left ( \frac{h^3}{12 \mu} \frac{\partial p}{\partial x}\right ) =  
 \frac{u}{2} \,\frac{\partial\, h\theta}{\partial x} + 
 \frac{\partial\, h\theta}{\partial t}
\label{eqpthetadim}
\end{equation}
under the complementarity conditions
\begin{equation}
\left \{
\begin{array}{l}
p~>~0 \qquad \Rightarrow \qquad \theta~ =~ 1 \\
\theta~<~1\qquad \Rightarrow \qquad p~=~0\\
0~\leq~\theta~\leq~1
\end{array}
\right.
\end{equation}
where $\mu$ is the viscosity of the lubricant. 

For the  contact we assume that the
lubricant-film thickness is known 
(and constant, equal to $d_{\mbox{\scriptsize lub}}$)
far upstream of the pad. This amounts to imposing
that $\theta = d_{\mbox{\scriptsize lub}}/h$ at the boundary of
the computational domain. More precisely, if the computational
domain corresponds to $x_\ell < x < x_r$, then assuming $u>0$
we impose $\theta(x_\ell,t) = d_{\mbox{\scriptsize lub}}/h(x_\ell,t)$.
%As already said, the boundary conditions along $x_2=0$ and
%$x_2=w$ are defined so as to enforce the proper periodicity in that direction.
An initial condition for $\theta$ is also provided.

At each instant the domain
spontaneously divides into a {\em pressurized} region, $\Omega^+$, where
$p>0$, and a {\em cavitated} region, $\Omega^0$, where the film is not
full ($\theta < 1$, see Fig. \ref{fig:domain}). At the boundary between $\Omega^+$ and
$\Omega^0$, the so-called cavitation boundary $\Sigma$, the
Elrod-Adams model automatically imposes mass-conservation.

\subsection{Pad dynamics and friction forces}

The dynamics of the pad is governed by the forces acting on it
along the $z$-direction. These forces are described below.

The {\em applied load} points downdwards (i.e.; along $-z$) and is 
assumed constant.
We denote by $W^{a}$ its value {\em per unit width}.

The {\em hydrodynamic force} originates from the pressure 
$p(x,t)$ that
develops in bearing. Its
value per unit width is given by
\begin{equation}
W^{h}(t)=\int_a^{b} p(x,t)~dx
\end{equation}

All lubricant films obtained 
in the simulations are thick enough to exclude the possibility of direct
contact between the pad and the runner.
Solid-solid forces are therefore neglected.

If $m$ is the mass of the pad per unit
width, the dynamical equation for the pad's vertical 
displacement reads
\begin{equation}
m \frac{d^2Z}{dt^2}  =  -W^{a}(t)+W^{h}(t)
\label{eqdynamics}
\end{equation}
This is supplemented with initial conditions for
$Z$ and $Z'$ at $t=0$. 

The friction force per unit width is given by
\begin{equation}
F = \int_a^{b}
\left ( \frac{\mu u\,g(\theta)}{h}+\frac{1}{2}h\frac{\partial p}{\partial x} + p\frac{\partial h_{L}}{\partial x}
 \,\right ) ~dx
\end{equation}
where the function $g(\theta)$ is taken as 
\begin{equation}
g(\theta) ~=~\theta\,s(\theta)
\end{equation}
with $s(\theta)$ the switch function 
\begin{equation}
s(\theta) = \left \{ \begin{array}{ll}
1 & \mbox{if}~~\theta > \theta_s \\
0 & \mbox{otherwise} \end{array}\right.
\end{equation}
In this friction model $\theta_s$ is a threshold for the onset
of friction, interpreted as the minimum lubricant fraction needed
for shear
forces to be transmitted from one surface to the other. In all
calculations the value $\theta_s = 0.95$ has been adopted. Choosing
another value for $\theta_s$ has been shown to not alter
the behavior of the contact significantly \cite{checo13}. Notice
from the definition of $g(\theta)$
that for $\theta > \theta_s$ the friction is calculated with
an ``effective viscosity'' equal to $\theta\,\mu$.
The friction coefficient then results from
\begin{equation}
f = \frac{F}{W^{a}}
\end{equation}

Remark: The term $p \frac{\partial h_L}{\partial x}$ is not a shear
force. Instead, it corresponds to
the projection of the pressure force along $x$ when the normal to the
liner is not along $z$. This term is omitted in the
literature, which mostly considers textures on the pad. Notice that
this term is necessary for the forces on the pad and on the runner
to be equal and opposite (action-reaction principle).


\subsection{Non-dimensionalization and final equations}

We consider a non-dimensionalization of the equations based on a
velocity scale $U$ (taken as equal to $u$), a length scale
$L$ (taken as equal to the pad's length), and a film thickness
scale $H$. No obvious candidate for $H$ is available because we will be 
comparing results with different pad's curvatures and different
texture depths, so that for now we leave $H$ unspecified. All computations are
non-dimensional.

The scales above, together with the viscosity $\mu$ of the lubricant,
lead to the following derived scales for the different
quantities

\begin{table}[htb]
\begin{center} \caption{Basic and derived scales}
\vspace{0.3cm}
\begin{tabular}{|l|l|l|}
\hline
Quantity & Scale & Name\\ \hline
& & \\
$x$,$\lambda$ & ${L}$ & horizontal coordinate, texture period\\
%& & &\\
$u$ & ${U}$ & relative velocity\\
%& & &\\
$R$ & ${L}$ & pad's profile curvature radius\\
%& & &\\
$t$ & $\frac{L}{U}$ & time\\
%& & &\\
$h$, $C$ & $H$ & gap thickness, minimum clearance\\
%& & &\\
$Z$,$d$ & $H$ & pad's vertical position, texture depth\\
%& & &\\
$p$ & $\frac{6\mu U L}{H^2}$ & pressure\\
%& & &\\
$W^{a}, W^h$ & $\frac{6\mu U L^2}{H^2}$ & applied and hydrodynamic forces per unit width\\
%& & &\\
$F$ & $\frac{\mu U L}{H}$ & friction force per unit width\\
%& & &\\
$m$ & $\frac{6 L^4 \mu}{H^3 U}$ & mass per unit width \\ & & \\\hline
\end{tabular}
\end{center}
\end{table}

Notice that, since the scales for radial and friction forces
are different, the friction coefficient is given by
\begin{equation}
f = \frac{H}{6L}~\frac{\hat{F}}{\hat{W}^{a}}
\end{equation}
where the carets (hats) denote the corresponding non-dimensional
quantity.

Upon non-dimensionalization of all variables, and 
omitting all carets for simplicity, 
the complete non-dimensional mathematical
problem to be solved reads:

``Find trajectory $Z(t)$, and fields $p(t)$, $\theta(t)$, defined
on $\Omega = (x_\ell,x_r)$, satisfying
\begin{equation}
\left \{
\begin{array}{l}
Z(0) ~ = ~ Z_{0}, \qquad Z'(0) ~ = ~ V_{0}, \\
\theta(x_\ell,t) = d_{\mbox{\scriptsize lub}}/h(x_\ell,t) \\
p~>~0 \qquad \Rightarrow \qquad \theta~ =~ 1 \\
\theta~<~1\qquad \Rightarrow \qquad p~=~0\\
0~\leq~\theta~\leq~1
\end{array}
\right.
\end{equation}
and
\begin{eqnarray}
m \frac{d^2Z}{dt^2}  &=&  -W^{a}+W^{h}(t)
\label{eqdynamics_a}\\
\frac{\partial }{\partial x} \left ( h^3 \frac{\partial p}{\partial x} \right ) &=&  
 u \,\frac{\partial\, h\theta}{\partial x} + 
 2\,\frac{\partial\, h\theta}{\partial t}
\label{eqptheta}
\end{eqnarray}
where
\begin{eqnarray}
h(x,t)&=& \left \{
\begin{array}{ll}
h_L(x-u\,t)+h_U(x)+Z(t)&\mbox{if}~~a < x < b\\
h_L(x-u\,t)+e& \mbox{otherwise}
\end{array} \right. ,\\
W^{h}(t)&=&\int_{a}^{b} p(x,t)~dx~,
\end{eqnarray}
and the functions $h_L(x)$ and $h_U(x)$ are known explicitly.''

The formula for the non-dimensional friction force per unit width is:
\begin{equation} \label{eq:adim-friction}
F = \int_a^{b}
\left ( \frac{\mu u\,g(\theta)}{h}+3h\frac{\partial p}{\partial x}
+6p\frac{\partial h_{L}}{\partial x}
 \,\right ) ~dx
\end{equation}
Notice the term $6\,p\,\frac{\partial h_L}{\partial x}$, 
which appears only when the texture moves across the domain
and is omitted in much of the literature.

\section{NUMERICAL METHOD AND DESCRIPTION OF NUMERICAL TESTS}

The numerical treatment is exactly that described in
Ausas et al\cite{ausas09}. %hugo: esa cita sale mal
It consists of a finite volume, mass-conserving
method with upwinding discretization of the Couette flux and
centered discretization of the Poiseuille flux, together with an iterative
imposition of the cavitation conditions ($p \geq 0$;
$\theta < 1\,\Rightarrow\,p=0$; $p>0\,\Rightarrow\,\theta=1$)
by means of a Gauss-Seidel-type algorithm. 
The dynamical equation (\ref{eqdynamics_a}) for
$Z(t)$ is discretized by a Newmark scheme, which is built into
the overall iterative process. 

Although the study focuses
on one-dimensional problems, it is only recently that 
numerical assessments as detailed as those reported in this
article have become affordable. 
Further, since we
consider the texture to be on the runner and thus moving through
the computational domain, the hydrodynamic simulation must
be run in unsteady mode, with a small enough time
step to capture the vertical oscillations of the pad as
textures pass under it. 

An effort has been made in this work to refine
the space and time discretizations until the results become
discretization-independent.
Very fine spatial meshes are needed for this,
because each texture cell needs
to be discretized into one hundred or more
finite volumes. The coarsest spatial mesh used consisted of $512$ 
cells along $x$, and some simulations were double-checked with up to $4096$
cells. Although the method is fully-implicit, a unit-CFL condition
was enforced, leading to time steps that ranged from $\Delta t = 4\times 10^{-3}$
to $\Delta t = 5 \times 10^{-4}$. 

The many thousands of computer runs compiled
in the next sections were made feasible by the
high robustness of the algorithm implemented in our in-house code\cite{ausas09}), %hugo: tambien
together with multigrid acceleration by the
Full Approximation Scheme\cite{fulton1986,venner2000}%hugo, tambien
and an effective shared-memory strategy.
The typical running time of one simulation
on the 512-cells mesh is about five minutes in a state-of-the-art desktop,
while about three hours are needed for the finest mesh.


The numerical experiments consider velocity and applied load as constant.
The runner's texture is assumed periodic, with period $\lambda$, i.e.;
$$
h_L(x) = h_L(x+\lambda)\qquad \qquad \forall \,x
$$
An initial transient takes place as the pad adjusts its average
vertical position. This transient lasts between 1 and 8 time units.
After it, the pad reaches a periodic vertical
motion, with time period $\lambda$ (notice that the non-dimensional
velocity is $u \equiv 1$), and all quantities become periodic.

The contacting surface of the pad is assumed to be a sector of
a circular cylinder of radius $R$, with $R$ varying between
$4$ and $1024$. Since the length of the pad is $1$,
radii $R$ greater than about $128$ correspond to essentially
conformal contacts. A value $R \simeq 40$ is typical of compression
rings in the automobile industry \cite{gad2012}.

Let us analyze on what non-dimensional parameters the solution of
the problem depends.
Since we focus on the periodic regime, the initial conditions for
$\theta$, $Z$ and $Z'$ are irrelevant. In this first study we
have decided not to address starvation effects. For this purpose,
$d_{\mbox{\scriptsize{lub}}}$ was chosen
large enough to assure fully-flooded conditions, thus making this
parameter also irrelevant.
There only remain two parameters
in the model: The mass per unit width $m$ and the applied load $W^a$.

The default values that we select for these parameters are
$$
m_0 = 2\times 10^{-5} \qquad \mbox{and} \qquad
W^a_0 = 1.666\times 10^{-4}
$$
Along the experiments, other values were tested so as to
assess the problem's sensitivity to the data.

When considering realistic scale values such as $U=10$ m/s, $L = 10^{-3}$ m,
$H = 1$ micron, $\mu = 4\times 10^{-3}$ Pa-s, the value selected
for $m_0$ corresponds to
$0.048$ kg$/$m, which is typical of compression rings in car engines.

The value selected for $W^a_0$, with the same scales as above,
corresponds to an applied load of 40 N/m, also in the range of
pre-stress loads of piston rings. A more popular non-dimensionalization
of the applied load is provided by the Stribeck number $S$,
defined by $S = \frac{\mu\,u}{W^{a}}$ (the quantities on the right-hand
side are dimensional this time). 
Back to non-dimensional variables, the previous relation 
in our non-dimensionalization becomes
\begin{equation}
S = \frac{H^2}{6\,L^2}~\frac{\hat{u}}{\hat{W}^{a}}
\label{stribeck2}
\end{equation}
so that $W^a_0$ corresponds to a Stribeck number $S_0 = 10^{-3}$.

Finally, let us define the main tribological quantities that will
be extracted from the numerical simulations.
The instantaneous friction coefficient $f(t)$ and clearance $C(t)$
depend periodically on time once the bearing has reached its periodic
regime. The tribological quantities that will be considered representative
of the periodic regime are the {\em average friction coefficient}
\begin{equation}
\overline{f} ~=~\frac{1}{\lambda}~\int_T^{T+\lambda}f(t)~dt
\end{equation}
which characterizes the power lost to friction,
and the {\em minimum clearance}
\begin{equation}
C_{\min}~=~\min_{T\leq t \leq T+\lambda}~C(t)
\end{equation}
which can be used to characterize wear. In the previous definitions
$T$ is large enough for the bearing to have reached its periodic regime
(i.e.; $T>8$).


\section{EFFECTS OF THE RUNNER'S TEXTURE}
\label{section4}

A large set of experiments was conducted in the simplified setting of 
one-dimensional transverse textures. Further, the runner was assumed to be sinusoidal
of depth $d$, namely
\begin{equation}
h_L(x)= d~\frac{1\,-\,\cos(2\,\pi\,x/\lambda)}{2}
\end{equation}
(notice that $h_L$ is a ``depth'' (positive downwards), so that it is minimal at
crests and maximal at troughs).
Such a runner was selected so as to observe the dynamics of the system
under maximal smoothness of the contacting surfaces. The applied
load was fixed at $W^a_0$ corresponding to a Stribeck number
$S=10^{-3}$. At this load, the equilibrium clearance of the pad for an
untextured runner depends on $R$ as shown in Table \ref{tableuntext}, which
also shows the friction coefficient for each $R$. The minimal friction
is obtained for $R = 8$, while the maximum clearance takes place for $R=16$.
Notice that for an untextured runner the equilibrium values of
$f$ and $C$ do not depend on $m$.

\begin{table} 
\begin{center}
\begin{tabular}{ccc}
\hline
 & & \\
$R$ & $f$ & $C_{\min}$ \\
& & \\
\hline
& & \\
4 &  $7.607 \times 10^{-2}$ & 6.369\\
8 &  $7.449 \times 10^{-2}$ & 7.809\\
16 &  $8.122\times 10^{-2}$ & 8.021\\
32 &  $9.562\times 10^{-2}$ & 7.408\\
64 &  $1.189\times 10^{-1}$ & 6.363\\
128 &  $1.536\times 10^{-1}$ & 5.231\\
256 &  $2.030\times 10^{-1}$ &  4.325\\
512 &  $2.612\times 10^{-1}$ & 3.695\\
$~~~~$ 1024$~~~~$ &  $~~~~2.871\times 10^{-1}~~~~$ & $~~~~$3.324$~~~~$\\
& & \\
\hline
\end{tabular}
\end{center}
\caption{Friction coefficient $f$ and clearance $C_{\min}$ for pad's with
several values of $R$ once they reached the equilibrium position
over an {\em untextured} runner. The computations were made
for a load $W^a = W^a_0$.}
\label{tableuntext}
\end{table}

\subsection{General description of the intervening phenomena}
\label{gendescrip}

The study was conducted for different values of the texture
parameters $d$ and $\lambda$, also considering different values of
the pad's curvature radius $R$. The mass of the pad was fixed
at $m=m_0$. Some selected
cases are discussed below (see Table \ref{tablecases} for details of
each case). Notice that some cases have rather large values of $\lambda$,
of the order of the pad's length. Though this could be viewed as
a ``waviness'' of the runner instead of as a texture, we keep the 
word ``texture'' for all values of $\lambda$.

\begin{table}
\begin{center}
\begin{tabular}{ccccccccc}
\hline
 & & & & & & & &\\
Case \# & $W^a$ & $R$ & $\lambda$ & $d$ & $\overline{f}$ & $C_{\min}$ & $V_f$ & $V_C$ \\
& & & & & & & & \\
\hline
& & & & & & & & \\
$~~~$1$~~~$ &  $~~1.66\times 10^{-4}~~$ &  $~~~$32$~~~$& $~~~$1$~~~$ & $~~~$5$~~~$ & $0.082$ & $6.89$ & -14\% & -7\%\\ %gustavo
& & & & & & & & \\
$~~~$2$~~~$ &  $~~1.66\times 10^{-4}~~$ & $~~~$32$~~~$& $~~~$0.1$~~~$ & $~~~$5$~~~$ & $0.114$ & $5.66$ & +19\% & -24\%\\
& & & & & & & & \\
$~~~$3$~~~$ &  $~~1.66\times 10^{-4}~~$ & $~~~$256$~~~$& $~~~$0.1$~~~$ & $~~~$5$~~~$ & $~~0.193~~$& $~~2.52~~$ & -5\% & -42\%\\
& & & & & & &  &\\
$~~~$4$~~~$ &  $~~1.66\times 10^{-4}~~$ & $~~~$256$~~~$& $~~~$1$~~~$ & $~~~$5$~~~$ & $~~0.084~~$& $~~6.33~~$ & -59\% & +46\%\\
& & & & & & & & \\
\hline
\end{tabular}
\end{center}
\caption{Details of the cases discussed in Sections \ref{section4}.
$W^a$ stands for the applied load, $m$ for the linear mass,
$R$ for the curvature radius of the pad, 
$\lambda$ for the texture's period (wavelength),
$d$ for its depth, $\overline{f}$ for the numerically obtained average
friction coefficient and $C_{\min}$ for the numerically obtained minimum
clearance. $V_f$ and $V_C$ stand for the relative variations of $\overline{f}$
and $C_{\min}$ with respect to the untextured case.
All quantities are non-dimensional. Notice that Case 4 is
introduced in subsection \ref{r256}.
}
\label{tablecases}
\end{table}


{\bf Case 1:} Consider first a ring with a moderate value of $R=32$, sliding on a
runner with a texture of period equal to the pad length ($\lambda = 1$) and 
depth $d=5$. The system behavior is then periodic in time with period equal
to one. Redefining $t=0$ as the instant at which the texture crest is exactly
under the left edge of the pad, profiles of $p$ and $\theta$ are shown in
Fig. \ref{figR32bubbles} for $t=0$, $0.25$, $0.5$ and $0.75$. 

At $t=0$ the crests of the texture coincide with the left ($x=0.5$)
and right ($x=1.5$) edges of the pad, while the trough coincides
with the pad's centerline ($x=1$). Notice that, {\em since the texture
is moving with the runner}, 
its {\em convergent} wedge corresponds to the sector between a crest
and its neighboring trough {\em to its right}, where the upward
normal is tilted {\em to the right}. This is the opposite to what occurs
on the pad's surface, in which the {\em convergent} wedge corresponds
to the left half of the pad, at which the (downward) normal is tilted
to the left.
Analogously, if the upward normal
to the runner is tilted to the left this corresponds to a {\em divergent}
wedge.

At $t=0$, thus, the
convergent wedge of the texture is located under the left half
of the pad, coincident with the convergent wedge of the pad's surface.
This leads to a pressure profile similar to that of the
untextured case, but much larger in value. At $t=0.25$ the pressure
peak has moved to the right, accompanying the texture's convergent wedge,
and the net effect is still that of increasing the average pressure under the pad.
At $t=0.5$ the texture's crest is under the pad's center, leaving the
left half of the pad over the texture's divergent wedge and consequently
at zero pressure. A cavitation bubble develops
there, which is evident from the $\theta$ profile ($\theta < 1$ implies
cavitation). This cavitation bubble,
which first appears at $t\simeq 0.304$ and $x \simeq 0.514$,
grows and moves to the right with the divergent wedge that
generates it, passing under the pad's center and eventually leaving the pad to
the right. In fact, this same cavitated region but corresponding to the
previous texture cell can be observed leaving the pad at $t=0.5$ (just
notice the region with $\theta < 1$ near $x=1.5$).
The cavitated region is always located at the {\em divergent}
wedge of the texture and travels with it. Its right boundary is a
{\em rupture} boundary, and as expected $\theta$ is continuous there
and $\partial p/\partial x=0$. The left boundary of the cavitated region
is a {\em reformation} boundary, and as such the $\theta$ profile is
very steep there, and $\partial p/\partial x$ exhibits a jump. 

{\em Remark:} It is worth pointing out that the profiles at $t=0$ 
show a cavitation boundary at $x \simeq 1.15$ for both the textured
and the untextured cases. This boundary is a {\em rupture} boundary
in the untextured runner and a {\em reformation} boundary
in the textured one.

{\bf Case 2:} With the same pad ($R=32$) and the same texture depth ($d=5$), it is
interesting to consider a texture size much smaller than the pad's length.
Figure \ref{figR32bubblesper01} shows profiles of $p$ and $\theta$ at
four instants of the periodic regime obtained with $\lambda = 0.1$.
Since the time period is equal to $\lambda$, and again defining $t=0$
when a crest is at the left edge of the pad, the instants shown in 
Fig.  \ref{figR32bubblesper01} correspond to $t=0$, $0.025$, $0.05$
and $0.075$. 

In this case the convergent wedge of the pad dominates the pressure
field in the left part of the contact. No cavitation bubbles form
there, and the effect of the texture is merely a modulation of the
pressure field with local maxima/minima at the convergent/divergent
wedges of the runner, respectively. The right part of the contact,
on the other hand, exhibits moving pressurized regions (local
pressure peaks) that are
absent in the untextured case. The texture's convergent wedges generate
locally pressurized regions, with cavitation bubbles between them, and
all this structure moves to the right with the runner. 

{\em Remark:} In fact, a careful analysis of the $\theta$ field
near the left edge of the pad shows that a tiny cavitation bubble
appears there when the divergent wedge of the texture enters the
contact, but it collapses soon afterwards and has no significant
effect on the pressure field. Cavitation bubbles that appear near
the left edge and collapse (or not) under the pad are discussed
further later on.

{\bf Case 3:} The last case we consider in this general description corresponds
to the same texture as before ($\lambda = 0.1$, $d=5$), but now
with a high-conformity pad ($R=256$). The corresponding profiles of
$p$ and $\theta$ at times $t=0$, $0.025$, $0.05$ and $0.075$ are shown
in Fig. \ref{figR256bubbles}. One observes that a cavitation bubble
develops whenever the divergent wedge of a texture enters the contact,
and that this bubble then travels with the texture until leaving the
pad at its right edge. Also, a locally pressurized region develops
at the convergent wedges of each texture and travels with it. The
height of the pressure peaks are slightly modulated by the pad's
shape, but there is no doubt that the overall solution is governed
by the runner's texture. The cavitation bubbles generated at each
texture impose a zero-pressure boundary condition for the pressurized
region that develop at the convergent texture wedges. The global pressure
field simply consists of the juxtaposition of these otherwise
independent local pressure peaks.

\begin{figure}[h]
     \begin{center}
	\scalebox{0.7}{\input{figs/1D_R32_lambda_1.pdf_t}}
    \end{center}
\caption{Instantaneous profiles of pressure $p$ (in red) and saturation $\theta$ (in blue) for a bearing with $R=32$, $d=5$ and $\lambda=1$ (Case 1 in the text) 
at $t=0$, 0.25, 
0.50 and 0.75 (from top to bottom) once the periodic regime has been attained. 
The steady pressure profile corresponding to the untextured runner 
is also plotted for comparison (in green).}  
\label{figR32bubbles}
\end{figure}


\begin{figure}[h]
     \begin{center}
	\scalebox{0.7}{\input{figs/1D_R32_lambda_01.pdf_t}}
    \end{center}
\caption{Instantaneous profiles of pressure $p$ (in red) and saturation $\theta$ (in blue) for a bearing with $R=32$, $d=5$ and $\lambda=0.1$ (Case 2 in the text) 
at $t=0$, 0.025, 
0.050 and 0.075 (from top to bottom) once the periodic regime has been attained. 
The steady pressure profile corresponding to the untextured runner 
is also plotted for comparison (in green).
}  
\label{figR32bubblesper01}
\end{figure}

\begin{figure}[h]
     \begin{center}
	\scalebox{0.7}{\input{figs/1D_R256_lambda_01.pdf_t}}
    \end{center}
\caption{
Instantaneous profiles of pressure $p$ (in red) and saturation $\theta$ (in blue) for a bearing with $R=256$, $d=5$ and $\lambda=0.1$ (Case 3 in the text) 
at $t=0$, 0.025, 
0.050 and 0.075 (from top to bottom) once the periodic regime has been attained. 
The pressure profile corresponding to the untextured runner 
is also plotted for comparison (in green).
}  
\label{figR256bubbles}
\end{figure}

\begin{figure}[h]
     \begin{center}
	\subfigure[][]{\scalebox{0.65}{\input{figs/map_R32_f_new.pdf_t}} \label{fig:R32f}}
	\subfigure[][]{\scalebox{0.65}{\input{figs/map_R32_C_new.pdf_t}} \label{fig:R32C}} \\
	\subfigure[][]{\scalebox{0.65}{\input{figs/map_R256_f_new.pdf_t}} \label{fig:R256f}}
	\subfigure[][]{\scalebox{0.65}{\input{figs/map_R256_C_new.pdf_t}} \label{fig:R256C}}	
    \end{center}
\caption{Relative differences \subref{fig:R32f} $V_{f}$ and  \subref{fig:R32C} $V_C$ (expressed in percentages) with respect to the untextured runner, as functions of the
texture parameters $d$ and $\lambda$ for a pad with $R$=32. The colorbars (as the isolines) indicate these percentages. Parts \subref{fig:R256f} and \subref{fig:R256C} of the
Figure show analogous plots for $R$=256. All simulations computed with $W^a=W^a_0$ and $m=m_0$. The specific cases 1 to 4 discussed in the text are shown as white dots.
}  
\label{fig:maps}
\end{figure}

\begin{figure}[h]
     \begin{center}
	\scalebox{0.7}{\input{figs/discont.pdf_t}}
    \end{center}
\caption{Profiles of $p$ and $\theta$ in the periodic regime corresponding to
bearings with $\lambda$=0.1, $R$=32, $m = m_0$ and $W^a=W^a_0$. The depth for the top 
graph is $d$=8.15 and for bottom one it is $d$=8.20.}  
\label{fig:disc}
\end{figure}

\begin{figure}[h]
     \begin{center}
	\scalebox{0.8}{\input{figs/disc_R32_Z.pdf_t}}
    \end{center}
\caption{$Z(t)$ for textured bearings with $\lambda$=0.1, $R$=32, $m=m_0$ and 
$W^a = W^a_0$. 
Noticed the dramatic change in behavior when $d$ changes from $8.15$ to $8.20$.} 
\label{fig:disc-Z}
\end{figure}


\begin{figure}[h]
     \begin{center}
	\scalebox{0.7}{\input{figs/1D_R256_lambda_1.pdf_t}}
    \end{center}
\caption{
Instantaneous profiles of pressure $p$ (in red) and saturation $\theta$ (in blue) for a bearing with $R=256$, $d=5$ and $\lambda=1$ (Case 4 in the text) 
at $t=0$, 0.025, 
0.050 and 0.075 (from top to bottom) once the periodic regime has been attained. 
The pressure profile corresponding to the untextured runner 
is also plotted for comparison (in green).
}  
\label{figR256bubblesper1}
\end{figure}



\begin{figure}[h]
     \begin{center}
	\subfigure[][]{\scalebox{0.62}{\input{figs/1D_R32_Z.pdf_t}}\label{fig:32-Z}} 
	\subfigure[][]{\scalebox{0.62}{\input{figs/1D_R256_Z.pdf_t}}\label{fig:256-Z}} \\	
    \end{center}
\caption{Ring position $Z(t)$ and friction force $F(t)$ (detail in figure) for rings with a curvature radius $R$ of \subref{fig:32-Z} 32 
\subref{fig:256-Z} 256. The texture parameters are $\lambda$=1.0 and $d$=5. Notice that (a) corresponds to Case 1 in the text, while (b) corresponds to Case 4.}  
\label{fig:cases1and4}
\end{figure}



\subsection{Friction and clearance charts}

In this section we explore parameter space for the selected configuration.
Thousands of runs were made varying $R$ between 4 and 1024,
$\lambda$ between 0.1 and 2, $d$ between 0 and 10, 
and $S=10^{-3}$ or $S=0.5\times 10^{-3}$.

\subsubsection{R=32}

Very little or no friction reduction appears for $R=16$ or smaller.
A bearing of $R=32$ can be thus considered of moderate conformity
for this configuration. Taking $m=m_0$ and $W^a=W^a_0$ the parameters
are those of which one specific example was discussed as Case 1
of Section \ref{gendescrip}. For the specific depth $d=5$ and
the specific period $\lambda=1$ we observed in Fig. 
\ref{figR32bubbles} the pressure and saturation fields. The average friction
coefficient $\bar{f}$ is 14\% smaller for Case 1 than for an %gustavo
untextured runner. This significant improvement is however
accompanied by an 7\% loss in minimum clearance, suggesting %gustavo
an increase in wear.

It is interesting to see these values in the context of values
obtained for all other possible textures, at least within some
ranges. For this purpose, we performed 2500 simulations spanning
the whole ranges of $d$ and $\lambda$ and plotted two-dimensional
contours in the $d-\lambda$ plane in Fig. \ref{fig:maps}. The quantities plotted are:
\begin{itemize}
\item The relative variation $V_f$ of the average friction coefficient,
defined as
$$
V_f(d,\lambda) = \frac{\bar{f}(d,\lambda)-f_{\mbox{\scriptsize{untextured}}}}
{f_{\mbox{\scriptsize{untextured}}}}
$$
\item The relative variation $V_C$ of the minimal clearance,
defined as
$$
V_C(d,\lambda) = \frac{C_{\min}(d,\lambda)-C_{\min,\mbox{\scriptsize{untextured}}}}
{C_{\min,\mbox{\scriptsize{untextured}}}}
$$
\end{itemize}
Please notice that the untextured case corresponds to the vertical
axis of the plots ($d=0$). 

In Fig. \ref{fig:maps}(a) a contour map of $V_f$ is shown.
The axis $d=0$ has obviously $V_f=0$, and one observes that,
depending on whether $\lambda > 0.5$ or not, $V_f$ becomes
negative or positive when the texture depth $d$ is increased.
For $\lambda > 0.5$ the friction diminishes as the texture
is made deeper. Relative diminutions of about 30\% can %gustavo
be seen at the top right corner of the chart, corresponding to
$d=10$ and $\lambda=2$. Interestingly, for some given depth
there exists an optimum period in terms of friction. It is
roughly of value $\lambda=1$ (texture's wavelength equal to the
pad's length), but it increases with $d$ to about $1.7$ for
$d=10$. Notice that Case 1 discussed in Section \ref{gendescrip}
lies close to the optimum period for $d=5$. From the
position on the diagram one may conclude that Case 1 is
indeed representative of the friction-reducing textures
for pads with $R=32$.

The contour map of $V_C$ shown in Fig. \ref{fig:maps}(b),
in turn, shows that all textures have minimal clearances smaller
than that obtained with an untextured runner. Only negative
values of $V_C$ are observed in the chart. 

An interesting
phenomenon occurs at the bottom right sector of Figs.
\ref{fig:maps}(a) and (b). Steep changes in $C_{\min}$
when $d$ is varied around $d=8$ are evident in the
sector $\lambda < 0.2$. Significant variations of
$\bar{f}$ are also observed there.  The physics behind
this phenomenon can be understood starting from
Case 2 in Section \ref{gendescrip}, which has $d=5$ and
$\lambda=0.1$. The texture-generated oscillations in the 
pressure field (see Fig. \ref{figR32bubblesper01}) are superposed
on the baseline pressurization curve generated by
the pad's geometry, of which the pressure profile 
generated on the untextured runner provides an estimate.
As $d$ is increased keeping the other parameters 
fixed at the values of Case 2, some of the pressure
oscillations touch the cavitation pressure and
bubbles appear near the left edge of the pad. For $d=8.15$
the $p$ and $\theta$ profiles are as shown in Figure \ref{fig:disc}
(top graph). A bubble is seen to have grown on the
left side of the pad. This cavitation bubble travels
to the right with the divergent microwedge that generated
it, but does not succeed to pass under the pad's center.
Along its travel to the right, at some instant it 
gets pressurized and collapses (the value of $\theta$
goes back to 1 and the pressure becomes positive when
the microwedge reaches $x \simeq 0.7$).
One other bubble is generated at the same divergent
wedge when it gets to $x \simeq 1.15$, as can be
seen in the figure, which then travels until the right
edge of the pad.

Between $d=8.15$ and $d=8.2$ a catastrophic event
(in the mathematical sense) takes place. Increasing
the texture depth makes the divergent microwedges steeper
and the cavitation bubbles created near the inlet now
travel under the pad without collapsing until the right edge.
The pressure field undergoes a major change, since
the relatively large pressurization region that existed
between $x=0.7$ and $x=1.15$ now has disappeared.
The pad looses lift and only re-encounters equilibrium
at a much lower position. All these features can be
seen in Fig. \ref{fig:disc} (bottom graph). In Fig. \ref{fig:disc-Z}
we plot the vertical movement of the pad as a
function of time when the pad starts from the
position $Z(0)=4$. For $d\leq 8.15$ it climbs to
values of $Z$ of about $4.5$, but for $d\geq 8.2$
it descends to about $2$ units from the $x$-axis.

This interesting phenomenon, which does not depend on
the model adopted for the friction force, 
suggests caution against carelessly increasing
the depth of the texture, since tribological performance
deteriorates substantially if this transition occurs.
Experimental investigations could bring light
onto the actual physical occurrence of this prediction of the
Elrod-Adams model. Cases of high sensitivity of the friction coefficient
to the texture depth have already been experimentally detected
by Scaraggi {\em et al} \cite{scar2013b}.


\subsubsection{R=256}
\label{r256}

Case 3 of Section \ref{gendescrip} has $R=256$, $d=5$
and $\lambda = 0.1$. It corresponds to the central point
of the horizontal axis in the graphs of $V_f$ and $V_C$
in Figs. \ref{fig:maps} (c) and (d). Again, 2500 simulations
were performed to compute the data for these graphs.

The value $R=256$ corresponds to a high-conformity contact,
in which the pressure field consists of a train of local
pressurized regions (one at each convergent microwedge)
traveling under the pad (see Fig. \ref{figR256bubbles}). 
Case 3 is an example of a
poorly designed texture, because increasing the period
$\lambda$ a little (to $0.4$ for example) would reduce
the friction by about 40\% while keeping the minimum
clearance at a value similar to that of the untextured
bearing. The best predicted performances in terms of
both friction and wear correspond to $\lambda\simeq 1.5$
and $d=10$.

In order not to depart too much from Cases 1-3 while
investigating the friction-reducing sector of the charts,
let us consider {\bf Case 4}, which has the
same runner as Case 1 but with the pad of Case 3; i.e.,
$$
\mbox{\bf Case 4: } R=256,\qquad d=5, \qquad \lambda=1
$$
For this bearing, the texture brings more than 60\%
improvement in friction together with 40\% increase
in minimum clearance, as compared to the untextured results.
Instantaneous plots of pressure and saturation at
$t=0$, $0.25$, $0.5$ and $0.75$ are shown in Fig.
\ref{figR256bubblesper1}. Instead of the train of ten bubbles
and ten pressure peaks
translating under the pad as it was in Case 3, one now 
essentially observes a
single cavitation bubble and a single pressurized region
during most of the period. The larger size of the instantaneous
pressurized region allows the pressure (and consequently the hydrodynamic
force) to attain values comparable to those of the untextured
case without diminishing (in fact, increasing) the clearance
with respect to the untextured value. A comparison of $Z(t)$
and $F(t)$ for Cases 1 and 4 can be found in Fig. \ref{fig:cases1and4},
where the complete evolution from an initial position $Z(0)=4$
is shown (the instantaneous friction force is depicted just
for one period). Notice that both cases, though having widely
different values of $R$, have similar average $F$.

\subsection{Increasing the load}

A high sensitivity to the load is not a positive characteristic
for a bearing. Even if the bearing works under constant
{\em dimensional} load and at constant velocity, 
the {\em non-dimensional} load $W^a$
is likely to vary in time because of its dependence on the 
viscosity of the lubricant. 

To explore this issue, additional charts of $V_f$ and
$V_C$, analog to those of Fig. \ref{fig:maps}, were computed
with twice the load, i.e., with $W^a=\,2\,W^a_0$. The results
can be found in Fig. \ref{fig:maps-halfs}. Let us compare part
(a) of the figure, corresponding to $V_f$ contours
for $R=32$, with its reference-load counterpart
of Fig. \ref{fig:maps}(a). One sees that doubling the load 
makes the friction-reduction region in the $d-\lambda$ plane
to shrink a little, but the global qualitative and quantitative
trends are preserved. The abrupt transition discussed in the
previous sections changes its location slightly, but otherwise
the bearing performance is not severely affected by the change
in load. Similar remarks can be made about the minimum clearance
examining part (b) of Fig. \ref{fig:maps-halfs}.

The weak sensitivity of $V_f(d,\lambda)$ to $W^a$ is also
evident from the chart corresponding to $R=256$, which
is found in Fig. \ref{fig:maps-halfs}(c). The chart of $V_C$,
in turn (Fig. \ref{fig:maps-halfs}(d)), in turn, exhibits
some more sensitivity to the load. The maximum relative gain in
$V_C$, which is more than 40\% for the reference load, reduces
to about 25\% for $W^a=\,2\,W^a_0$. The beneficial effect of
the texture of Cases 1 and 4 is nevertheless maintained at
this larger load.



%%%%%%%%%%%%%%%%%%%%%%%%%%%%%%%%%%%%%%%%%%%%%%%%%%%%%%%%%%%%%%%%%%%%%%







\begin{figure}[h]
     \begin{center}
	\subfigure[][]{\scalebox{0.65}{\input{figs/map_R32_f_2c_new.pdf_t}} \label{fig:R32f2c}}
	\subfigure[][]{\scalebox{0.65}{\input{figs/map_R32_C_2c_new.pdf_t}} \label{fig:R32C2c}} \\
	\subfigure[][]{\scalebox{0.65}{\input{figs/map_R256_f_2c_new.pdf_t}} \label{fig:R256f2c}}
	\subfigure[][]{\scalebox{0.65}{\input{figs/map_R256_C_2c_new.pdf_t}} \label{fig:R256C2c}}	
    \end{center}
\caption{Same organization of results as in Fig. \ref{fig:maps}, but changing the
applied load to $W^a = 2\,W^a_0$.
 }  
\label{fig:maps-halfs}
\end{figure}





\section{DISCUSSION OF RESULTS}


For low-conformity bearings ($R/L \leq 16$) the results above confirm
that no beneficial effects in either friction or wear are predicted
by the model, which is consistent with previous 
findings \cite{gad2012,checo13,etsion2013}.

For moderate- ($R/L = 32$) and high-conformity ($R/L = 256$) bearings,
on the other hand,
the two-dimensional contour plots
of friction and clearance
together with the specific simulations (cases 1-4) selected
for detailed scrutiny, revealed some {\em general} trends and underlying physical
mechanisms:
\begin{enumerate}
\item Friction coefficients of high-conformity bearings are greater
than those of moderate-conformity ones. In both cases, textures of
period comparable to the pad's length and sufficiently deep 
seem to be optimal in terms of friction for the
ranges considered. The mathematical model predicts a
relative reduction of friction
of up to 40\% for moderate-conformity bearings and up to 75\% for
high-conformity ones.
\item The optimal period (in terms of friction and wear)
for each depth $d$ is predicted to be comparable to the 
pad's length ($\lambda \simeq 1$) and a slowly
increasing function of $d$.
\item Though textures can indeed reduce the friction of moderate-conformity
bearings, all textured runners produce pad-to-runner clearances
that are smaller than that of the untextured runner.
\item Suitable textures significantly
improve the clearance of high-conformity bearings (by 20\% or more), 
thus predicting reductions in wear.
\item  The basic mechanism of friction reduction 
is a local pressurization of the convergent
microwedges at each texture cell, accompanied by local cavitation
at the divergent microwedges. The cavitation bubble that forms
at each texture cell prevents the appearance of
local negative pressure peaks that would cancel out the positive
lift force generated at the convergent microwedges. The extra lift
generated in this way increases the clearance and reduces friction.
\item This explains why longer texture periods (wavelengths of order $L$) 
are predicted to be more efficient than shorter ones, since the capacity of
a wedge to generate lift increases with length. 
\item Best performance is achieved when the depth is of the order of
twice the pad's fly height $Z$, so that the convergent microwedge (which
has film thickness $Z+d$ at the trough and $Z$ at the crest, approximately) 
has taper ratio $(Z+d)/Z$ in the range $2.5 - 3.5$. Taper ratios in this range
are known to be effective not only for generating load capacity (the
Rayleigh step, which is optimal, has a slightly smaller taper ratio of 1.866) but also
for minimizing friction\cite{rahmani09}. 
\begin{figure}[h]
     \begin{center}
     {\scalebox{1.0}{\input{figs/d_lambda_new.pdf_t}}}
    \end{center}
\caption{Texture depth corresponding to minimal friction $d_f(\lambda)$
as a function of the period $\lambda$. Shown are numerical data 
corresponding to $R=32$ and $R=256$, in each case considering
loads $W^a=W^a_0$, $W^a=2\,W^a_0$ and $W^a=3\,W^a_0$.}  
\label{fig:dflambda}
\end{figure}
\begin{figure}[h]
     \begin{center}
     {\scalebox{1.0}{\input{figs/tf_lambda_new.pdf_t}}}
    \end{center}
\caption{Taper ratio corresponding to minimal friction $t_f(\lambda)$
as a function of the period $\lambda$. Shown are numerical data 
corresponding to $R=32$ and $R=256$, in each case considering
loads $W^a=W^a_0$, $W^a=2\,W^a_0$ and $W^a=3\,W^a_0$.}  
\label{fig:tflambda}
\end{figure}
\item To prove the previous assertion, let us present
the results in Figs. \ref{fig:maps} and \ref{fig:maps-halfs} in
another way. For each period $\lambda$ there is a depth $d_f(\lambda)$
that minimizes friction, which is shown in Fig. \ref{fig:dflambda}.
For $R=256$, the minimal-friction depth $d_f(\lambda)$ is zero,
corresponding to the untextured runner, only for periods smaller
than $0.15$. For $\lambda > 0.15$ one observes $d_f(\lambda)$
increasing steadily with $\lambda$. A similar trend is observed
for $R=32$, though in this case the untextured runner is
optimal until $\lambda$ is about 0.5 (depending on the charge).
There is also a corresponding minimal clearance,
$C_{\min}(d_f(\lambda),\lambda)$, with can be taken as a representative
value for the pad's fly height. One can thus plot the taper ratios
corresponding to minimal friction for each $\lambda$, which
is given by
\begin{equation}
t_f(\lambda)=\frac{C_{\min}(d_f(\lambda),\lambda)+d_f(\lambda)}{C_{\min}(d_f(\lambda),\lambda)}
\label{eq:tflambda}
\end{equation}
This is done in Fig. \ref{fig:tflambda} for $R=32$ and $R=256$ and
for loads $W^a=W^a_0$, $W^a=2\,W^a_0$ and 
$W^a=3\,W^a_0$. Consider first the curve corresponding to $R=256$
and $W^a=W^a_0$. The taper ratio that yields minimal friction is
remarkably constant from $\lambda=0.2$ up to $\lambda=1.4$,
taking values in the range 2.5--3.5. As the load is increased,
and although $d_f(\lambda)$ decreases accordingly (see Fig. \ref{fig:dflambda}),
the optimal taper ratio remains in the same range (2.5--3.5) but
up to a smaller value of $\lambda$ (up to $\lambda=1.2$ for
$W^a=2\,W^a_0$ and up to $\lambda=1$ for $W^a=3\,W^a_0$). One
thus concludes that for highly-conformal bearings the Elrod-Adams
model suggests selecting the texture depth so that the taper ratio
is between 2.5--3.5, which corresponds to $d$ in the range
between $1.5\,C_{\min}$ and $2.5\,C_{\min}$.
\item For the case $R=32$ one observes that the untextured runner
is optimal up to $\lambda\simeq 0.5$ (in fact, 0.45 for 
$W^a=W^a_0$ and 0.55 for $W^a=3\,W^a_0$). At $\lambda \simeq 0.5$ 
the optimal taper ratio $t_f(\lambda)$
undergoes a jump to a value of about 2.5 and from there increases
steadily with $\lambda$, remaining below 3.5 up to $\lambda=1$.
The model thus again suggests to take $d$ in the range
$1.5\,C_{\min}-2.5\,C_{\min}$, though only for texture periods larger
than half the pad's length this time.
\item Since $d/C_{\min}$ must be $\simeq 1.5 - 2.5$ to be effective, larger loads 
require shallower textures. This puts forward the challenge in 
designing textures for variable loads.
\item An interesting behavior is predicted for textures
with short wavelength ($\lambda < 0.25$). As the depth $d$
is increased, cavitation bubbles begin to form at the left
edge of the pad over each divergent microwedge. Each of these bubbles
travels with the wedge that generated it, but it collapses under
the pressurization effect of the converging part of the pad.
When the wedge gets to the divergent part of the pad, a new bubble
forms on it and travels with it until reaching the right edge of
the pad. This leaves a central portion of the pad pressurized.
\item For moderate-conformity bearings, the previous behavior
persists until the depth reaches a limit value of about $d=8$.
If $d$ is further increased, the bubbles no longer collapse
under the pad and the central pressurized portion of the pad
no longer exists. As a consequence, the clearance is severely
reduced (by up to 50\%!). This sudden transition should be
further investigated by experimental techniques, since our
results are model dependent.
\item For high-conformity bearings a similar transition occurs,
but at a much lower value of $d$.
\end{enumerate}


\section{Conclusions}

An extensive study has been reported on the effect of transverse
sinusoidal textures on the tribological performance of an
infinitely-wide thrust bearing with the texture on the runner. 
Contacts with different conformity were considered by varying
the ratio $R/L$, with $R$ the curvature radius of the pad and
$L$ its length. The analysis method consists of time- and mesh-resolved
simulations (with up to 4096 cells in the longitudinal direction
and 40000 time steps) with a finite volume approximation of the
Elrod-Adams model. 

Upon non-dimensionalization, the problem depends
on the mass of the pad $m$ and the load applied on it, $W^a$. For these
variables values representative of piston ring/liner contacts were
assumed. The remaining two free parameters are those defining the
texture: its depth $d$ and its period $\lambda$. More than ten thousand
simulations were run to construct two-parameter frictional and
clearance charts for the ranges $0\leq d\leq 10$ and $0.1\leq \lambda\leq 2$,
and some selected cases were subject to detailed scrutiny.

The analysis of these simulations confirmed that textures are predicted to be
beneficial only for contacts with moderate or high conformity ($R/L \geq 32$).
The mechanism involved in friction reduction was identified
as the local pressurization of the convergent wedge present in each
texture cell, in agreement with the mechanism proposed earlier 
for {\em stationary} textures \cite{ets05,dobrica2010} (corresponding 
to a textured pad in our setting). The bearing's response to the texture, 
at least as predicted by Elrod-Adams model, indicates that best
performance is obtained with
texture lengths comparable to the pad's length, and with depths
of approximately twice the pad-to-runner clearance. Though these conclusions
were drawn considering just one value of the pad's mass ($m=m_0$), it was
confirmed that they remain valid for twice and half this value ($m=2m_0$ and $m=m_0/2$),
though this complementary study was not included here for the sake of brevity.
Other general observations were collected and discussed in Section 5. 

Further numerical investigations extending the ranges of the
study reported here can of course unveil new phenomena, which of course
would depend on the physical validity of the adopted model. In fact, our
numerical assessment suggests as possible validation 
the investigation of
a sudden transition in clearance that is predicted as the texture depth
is increased under specific operating conditions.




\section{Acknowledgments}

This work was supported by Coordena\c{c}\~ao de Aperfei\c{c}oamento
de Pessoal de N{\'\i}vel Superior [grant number DS-8434433/M] (Brazil),
by Funda\c{c}\~ao de Amparo \`a Pesquisa do Estado de S\~ao Paulo
[grants numbers 2012/14481-8, 2011/24147-5] (Brazil),
by Conselho Nacional de Desenvolvimento Cient{\'\i}fico e Tecnol\'ogico
[grant number 308728/2013-0] (Brazil) and by Renault (France).

\bibliographystyle{vancouver}

\begin{thebibliography}{10}

\bibitem{dobrica2010}
Dobrica MB, Fillon M, Pascovici MD, Cicone T.
\newblock {Optimizing surface texture for hydrodynamic lubricated contacs using
  a mass-conserving numerical approach}.
\newblock Proc IMechE. 2010;224:737--750.

\bibitem{gad2012}
Gadeschi GB, Backhaus K, Knoll G.
\newblock {Numerical analysis of laser-textured piston-rings in the
  hydrodynamic lubrication regime}.
\newblock ASME Journal of Tribology. 2012; 134:041702--1--041702--8.

\bibitem{etsion2013}
Etsion I.
\newblock {Modeling of surface texturing in hydrodynamic lubrication.}
\newblock Friction. 2013; 3:195--209.

\bibitem{checo13}
Checo H, Ausas R, Jai M, Cadalen JP, Choukroun F, Buscaglia G.
\newblock {Moving textures: Simulation of a ring sliding on a textured liner}.
\newblock Tribology International. 2014; 72:131--142.

\bibitem{buscaglia05c}
Buscaglia GC, Ciuperca I, Jai M.
\newblock {The effect of periodic textures on the static characteristics of
  thrust bearings}.
\newblock ASME Journal of Tribology. 2005; 127:899--902.

\bibitem{buscaglia07}
Buscaglia G, Ciuperca I, Jai M.
\newblock {On the optimization of surface textures for lubricated contacts}.
\newblock Journal of Mathematical Analysis and Applications.
  2007; 335:1309--1327.

\bibitem{costa2007}
Costa HL, Hutchings IM.
\newblock {Hydrodynamic lubrication of textured steel surfaces under
  reciprocating sliding conditions}.
\newblock Tribology International. 2007; 40:1227--1238.

\bibitem{koval2011}
Kovalchenko A, Ajayi O, Erdemir A, Fenske G.
\newblock {Friction and wear behavior of laser textured surface under
  lubricated initial point contact}.
\newblock Wear. 2011; 271:1719--1725.

\bibitem{yin2012}
Bifeng Y, Li X, Fu Y, Yun W.
\newblock {Effect of laser textured dimples on the lubrication performance of
  cylinder liner in diesel engine}.
\newblock Lubrication Science. 2012; 24:293--312.

\bibitem{tomanik2013}
Tomanik E.
\newblock {Modelling the hydrodynamic support of cylinder bore and piston rings
  with laser textured surfaces}.
\newblock Tribology International. 2013; 59:90--96.

\bibitem{scaraggi2013}
Scaraggi M, Mezzapesa FP, Carbone G, Ancona A, Tricarico L.
\newblock {Friction properties of lubricated laser-microtextured-surfaces: an
  experimental study from boundary to hydrodynamic lubrication.}
\newblock Tribology Letters. 2013; 49:117--125.

\bibitem{qiu11}
Qiu Y, Khonsari M.
\newblock {Experimental investigation of tribological performance of laser
  textured stainless steel rings}.
\newblock Tribology International. 2011; 44:635--644.

\bibitem{grabon2013}
Grabon W, Koszela W, Pawlus P, Ochwat S.
\newblock {Improving tribological behavior of piston ring-cylinder liner
  frictional pair by liner surface texturing.}
\newblock Tribology International. 2013; 61:102--108.

\bibitem{cross2013}
Cross AT, Sadeghi F, Rateick RG, Rowan S.
\newblock {Hydrodynamic pressure generation in a pocketed thrust bearing.}
\newblock Tribology Transactions. 2013; 56:652--662.

\bibitem{zhang2012}
Zhang J, Meng Y.
\newblock {Direct observation of cavitation phenomenon and hydrodynamic
  lubrication analysis of textured surfaces}.
\newblock Tribology Letters. 2012; 46:147--158.

\bibitem{ausas2013}
Ausas RF, Jai M, Ciuperca IS, Buscaglia GC.
\newblock {Conservative one-dimensional finite volume discretization of a new
  cavitation model for piston-ring lubrication}.
\newblock Tribology International. 2013; 57:54--66.

\bibitem{buscaglia2013}
Buscaglia GC, Ciuperca I, Dalissier E, Mohammed J.
\newblock {A new cavitation model in lubrication:the case of two-zone
  cavitation.}
\newblock Journal of Engineering Mathematics. 2013; 83:57--79.

\bibitem{ausas07}
Ausas R, Ragot P, Leiva J, Jai G M~Bayada, Buscaglia G.
\newblock {The impact of the Cavitation model in the Analysis of Micro-Textured
  Lubricated Journal bearings.}
\newblock {ASME Journal of Tribology}. 2009; 129:868--875.

\bibitem{ausas09}
Ausas R, Jai M, Buscaglia G.
\newblock {A Mass-Conserving Algorithm for Dynamical Lubrication Problems With
  Cavitation}.
\newblock {ASME Journal of Tribology}. 2009; 131:031702-1--031702-7.

\bibitem{fowell07}
Fowell M, Olver A, Gosman A, Spikes H, Pegg I.
\newblock {Entrainment and Inlet Suction: Two Mechanisms of Hydrodynamic
  Lubrication in Textured Bearings}.
\newblock ASME Journal of Tribology. 2007; 129:336--347.

\bibitem{elrod1}
Elrod HG, Adams M.
\newblock {A computer program for cavitation. {T}echnical report 190.}
\newblock {1st LEEDS LYON Symposium on Cavitation and Related Phenomena in
  Lubrication, IME}. 1974; 103:354.

\bibitem{fulton1986}
Fulton SR, Ciesielsky PE, Schubert WH.
\newblock {Multigrid methods for elliptic problems: a review.}
\newblock Monthly Weather Review. 1986; 114:943--959.

\bibitem{venner2000}
Venner CH, Lubrecht AA.
\newblock {Multilevel methods in lubrication}.
\newblock Elsevier; 2000.

\bibitem{scar2013b}
Scaraggi M, Mezzapesa F, Carbone G, Ancona A, Sorgente D, Lugarà P.
\newblock {Minimize friction of lubricated laser-microtextured-surfaces by
  tuning microholes depth}.
\newblock Tribology International. 2014; 75:123--127.

\bibitem{rahmani09}
Rahmani R, Shirvani A, Shirvani H.
\newblock {Analytical analysis and optimization of the Rayleigh step slider
  bearing}.
\newblock Tribology International. 2009; 42:666--674.

\bibitem{ets05}
Etsion I.
\newblock {State of the art in laser surface texturing}.
\newblock ASME Journal of Tribology. 2005;127:248--253.

\end{thebibliography}

\include{fcaption}

\end{document}
% $Id: amcapaper.tex,v 1.23 2006/08/14 16:58:45 mstorti Exp $


